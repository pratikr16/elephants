\documentclass[10pt,twocolumn]{paper}
\usepackage{lmodern}
\usepackage{amssymb,amsmath}
\usepackage{ifxetex,ifluatex}
\usepackage{fixltx2e} % provides \textsubscript
\ifnum 0\ifxetex 1\fi\ifluatex 1\fi=0 % if pdftex
  \usepackage[T1]{fontenc}
  \usepackage[utf8]{inputenc}
\else % if luatex or xelatex
  \ifxetex
    \usepackage{mathspec}
  \else
    \usepackage{fontspec}
  \fi
  \defaultfontfeatures{Ligatures=TeX,Scale=MatchLowercase}
\fi
% use upquote if available, for straight quotes in verbatim environments
\IfFileExists{upquote.sty}{\usepackage{upquote}}{}
% use microtype if available
\IfFileExists{microtype.sty}{%
\usepackage[]{microtype}
\UseMicrotypeSet[protrusion]{basicmath} % disable protrusion for tt fonts
}{}
\PassOptionsToPackage{hyphens}{url} % url is loaded by hyperref
\usepackage[unicode=true]{hyperref}
\hypersetup{
            pdftitle={Paper summaries},
            pdfborder={0 0 0},
            breaklinks=true}
\urlstyle{same}  % don't use monospace font for urls
\usepackage[margin=2cm]{geometry}
\setlength{\emergencystretch}{3em}  % prevent overfull lines
\providecommand{\tightlist}{%
  \setlength{\itemsep}{0pt}\setlength{\parskip}{0pt}}
\setcounter{secnumdepth}{0}
% Redefines (sub)paragraphs to behave more like sections
\ifx\paragraph\undefined\else
\let\oldparagraph\paragraph
\renewcommand{\paragraph}[1]{\oldparagraph{#1}\mbox{}}
\fi
\ifx\subparagraph\undefined\else
\let\oldsubparagraph\subparagraph
\renewcommand{\subparagraph}[1]{\oldsubparagraph{#1}\mbox{}}
\fi

% set default figure placement to htbp
\makeatletter
\def\fps@figure{htbp}
\makeatother

\usepackage{libertine}

\title{Paper summaries}
\date{}

\begin{document}
\maketitle

\section{Paper summaries}\label{paper-summaries}

\begin{enumerate}
\def\labelenumi{\arabic{enumi}.}
\tightlist
\item
  \textbf{Valls Fox (2015)} : Elephants behave as central place foragers
  with multiple foci, periodically returning to water. On longer trips,
  elephants travel farther from water, as would be expected if speed
  were constant. In the dry season, the frequency of water trips is
  increased, as is the distance travelled from water, indicating
  increased movement rate. Simultaneously, elephants select for areas of
  low waterhole density, especially later in the dry season. This may
  represent selection for areas with lower densities of other, competing
  animals. This is the closest any work has reached to the present
  study, and posits thermoregulation as the driver of elephant recursion
  to water sources. Is not yet published (\emph{pers. comm.}), will
  serve as a strong supplementary work in the intro and discussion.
\item
  \textbf{Kinahan \emph{et al.} (2007)} : GPS loggers on elephants
  combined with landscape level temperature measurements from weather
  stations show that elephants select for thermally stable landscapes,
  ie, with lower rates of temp. change and also for shade. Landscape use
  may be thermally constrained in the species.
\item
  \textbf{Beest \emph{et al.} (2012)} : Moose \emph{Alces alces} are
  among the largest temperate ungulates and experience both heat and
  cold stress. In both winter and summer, moose experience temperatures
  above critical \texttt{thresholds}, and seek refuge in landscapes with
  cover, avoiding open areas. Used GPS + temp loggers on moose.densities
  Could form part of a compare/contrast with elephants given large size
  yet very different environment.
\item
  \textbf{Cain \emph{et al.} (2012)} : Zebra \emph{Equus quagga} and
  sable antelope \emph{Hippotragus niger} in Kruger NP show different
  movements in relation to perennial water sources during the dry
  season. Sable need to drink less frequently, and are able to occupy
  habitats further from water. While this results in higher travel costs
  between forage and water, and lost foraging time, it likely reduces
  predation risk at crowded waterholes. Used GPS loggers.
\item
  \textbf{Hirst (1975)} : Monograph on the ecology of African savanna
  ungulates. Useful for intro.
\item
  \textbf{Owen-Smith \& Goodall (2014)} : Three coexisting ungulate
  species along a body-mass gradient were GPS tagged in Kruger to
  examine how they coped with changes in rsource availability and
  predation risk related to seasonality. Buffalo \emph{Syncerus caffer},
  the largest of the three, showed the strongest reduction in activity
  levels with rising temps (proxied by time, assuming max temps at
  midday), but maintained similar activity levels between day and night.
  Smaller and more at-risk sable and zebra showed more diurnal than
  nocturnal activity. During the dry season, both also moved and foraged
  more than buffalo (lower energy deposits to draw from?). Zebra foraged
  longer than the others due to their physiology. Sable in a wet habitat
  showed similar activity levels as in a dry one, but moved more slowly,
  and did not increase the time spent foraging, likely due to better
  forage availability.
\item
  \textbf{Fuller \emph{et al.} (2014)} : Review putting forward the idea
  that water stress leads to increased animal body temps, and that this
  is the condition under which heterothermy occurs, with large mammals
  capable of maintaining \(T_b\) within a narrow range in arid habitats
  given access to food and water. Useful for intro, prompts question
  whether elephant heterothermy is seasonal.
\item
  \textbf{Leggett (2010)} : Desert dwelling elephant males in Namibia
  were GPS tracked for one year. Daily movement and movement rate were
  higher in the wet season than in the dry seasons. Peak movement times
  shifted with the season, with elephants moving faster early in the day
  in the hot dry season. Useful for intro and discussion.
\item
  \textbf{Giotto \emph{et al.} (2015)} : Asiatic wild ass \emph{Equus
  hemionus} movement in an arid region reveals a near 24h periodicity in
  revisits to sites. These sites hold resources in the forms of water,
  forage, and suitable topography. Territoriality was a social factor
  that appeared to bear upon revisit rates. Used GPS tags, useful for
  intro.
\item
  \textbf{Bennitt (2014)} : Fifteen buffalo collared with GPS loggers in
  the Okavango delta showed contrasting selection for proximity to water
  sources in the wet season, and areas close to water in the dry
  (late-flood) season. This contrasts with smaller sable antelope (Cain
  \emph{et al.} 2012), which spend more time away from waterholes in the
  dry season, and is related to larger animals' need to remain close to
  water to thermoregulate, either by drinking or bathing (Owen-Smith \&
  Goodall 2014).
\item
  \textbf{Hetem \emph{et al.} (2012)} : Arabian oryx \emph{Oryx
  leucoryx} were fitted with GPS loggers and black-globe temp loggers,
  allowing for the analysis of microclimate selection in the Arabian
  desert. Oryx selected cooler microclimates, presumably shade, than was
  found in the sun in all seasons. Oryx activity patterns shifted from
  largely crepuscular in the cooler months to being mostly nocturnal in
  the summer months. Useful for intro/disc. Study similar to current
  work, lacking only waterholes.
\item
  \textbf{Hetem \emph{et al.} (2010)} : Arabian oryx implanted with
  temperature loggers showed marked heterothermy in the summer, with the
  amplitude of this heterothermy also significantly higher in the warm
  dry than in the warm wet season. This points to water-stress as the
  primary driver of heterothermy in large mammals. Useful for
  intro/disc.
\item
  Some general articles linking climate change, especially warming, with
  species' habitat and distribution changes: Walther \emph{et al.}
  (2002), and Parmesan (2006).
\item
  Bowyer \& Kie (2009) : Black-tailed deer \emph{Odocoileus hemionius
  columbianus} select for stands of live oak on warm, windless, dry days
  in winter to avoid excess heat gain. Used VHF radio-telemetry and
  static weather sensors.
\item
  Aublet \emph{et al.} (2009) : Ibex \emph{Capra ibex} show a change in
  behaviour in summer, with older (and larger) males showing the most
  change. Peak foraging activity shifts to earlier in the day, probably
  to avoid high temps and sunshine.
\item
  Some papers on elephant movement and reliance on water: Boettiger
  \emph{et al.} (2011), Redfern \emph{et al.} (2003), Redfern (2002),
  Tshipa \emph{et al.} (2017), and Loarie \emph{et al.} (2009).
\item
  Schmidt \emph{et al.} (2016) : Muskoxen \emph{Ovibos moschatus} are
  non-migratory and are restricted to tundra habitats, encountering both
  heat and cold stress. Muskox movement increases with increasing
  temperature in summer, but increases with decreasing temperature in
  winter. Linearity of movement increases with increasing temperatures
  in winter, but decreases with rising temps in summer. These are very
  early results and require more context. Used GPS loggers and weather
  data from stations.
\item
  Lowe \emph{et al.} (2010) : Moose in Canada were shown to \emph{not}
  make use of thermal cover in the form of confierous tress, but rather
  to occupy the landscape as though unaffected by thermal stress. Used
  GPS collars and static weather data.
\item
  Stelzner (1988) : Yellow baboons \emph{Papio cynocephalus} probability
  of being found in shaded or unshaded areas was unaffected by heat
  stress (proxied by temp/rad metrics). However, the speed of movement
  through habitats offering more cover was lower than open areas. At the
  behavioural level, baboons spent more time resting in shaded areas
  than in open ones, leading to a cumulative slowing of the troupe when
  in woodland.
\item
  Cain \emph{et al.} (2006) : A review of thermoregulation in N. Am.
  desert ungulates, with refs to studies in other species from
  Asia/Africa.
\item
  Johnson \emph{et al.} (2002) : Ungulate movements (in caribou,
  \emph{Rangifer tarandus caribou}) are scale dependent, and intra- and
  inter-patch movements can be differentiated from tracking data, and
  also carry different costs, such as predation risks. Used GPS data.
\item
  Coughenour (2008) : A general chapter on the causes and consequences
  of the movement of large herbivores through landascapes. Might be
  useful for intro.
\item
  Rahimi \& Owen-Smith (2007) : Two herds of sable antelope in Kruger
  tagged with GPS transmitters showed increasing daily displacement
  during the dry season due to watering trips every few days. Nocturnal
  activity increased during this time relative to the rest of the year.
\item
  Nowack \emph{et al.} (2013) : African lesser bushbabies \emph{Galago
  moholi} use behavioural thermoregulation rather than physiological
  (torpor) in winter to conserve heat by increasing the intake of
  high-quality food, using shelter, reducing nocturnal activity, and
  increasing huddling. This allows them to remain normothermic rather
  than slip into heterothermy. Used radio-telemetry and static temp/RH
  loggers. May be useful for intro.
\item
  Fuller \emph{et al.} (2005) : Eight springbok \emph{Antidorcas
  marsupialis} fitted with internal temperature loggers showed that
  heterothermy is scale dependent; while 24h temps track ambient temps,
  mean daily temps show little variability between seasons, and poor
  response to ambient temps.
\item
  Weissenböck \emph{et al.} (2012) : Evidence for heterothermy in Asian
  elephants \emph{Elephas maximas}. Useful for the intro/discussion;
  used internal temp loggers, elephants tame.
\item
  Cain III \emph{et al.} (2008) : Desert living bighorn sheep showed no
  changes in their movement patterns in response to the removal of
  artificial(?) water sources. This may have been because of an unusual
  amount of precipitation which offset the lack of waterholes. Used GPS
  loggers, useful for discussion.
\item
  De Beer (2008) : Elephant home range size is determined by landscape
  heterogeneity and by waterhole density. Useful for intro/discussion.
\item
  Cromhout (2007) : Buffalo space use in S.Af. in different seasons,
  includes the idea of nutritional stress in the dry season, but shows
  that buffalo in the dry season have larger ranges, contradicting
  Bennitt (2014).
\item
  Hetem \emph{et al.} (2007) : The use of a small black globe mounted on
  the dorsal side of a GPS collar can yield good approximations of the
  thermal microclimate of an animal. Animal use of thermally favourable
  microhabitats shown in three ungulates as validation of the device.
\item
  Shrestha \emph{et al.} (2012) : Further evidence that thermal states
  in antelope (eland \emph{Taurotragus oryx}, blue wildebeest, and
  impala) are likely to be dependent on access to water and nutrition.
\item
  Shrestha (2012) : Thesis on antelope in S.Af., includes Shrestha
  \emph{et al.} (2012), may be useful for intro.
\item
  Shrestha \emph{et al.} (2014) : Acitivity of larger ungulates is
  reduced in the hot dry season, more than that of smaller ones, a trend
  which continues into spring. Used black globe temperatures as a
  predictor for activity, with an observed negative correlation. Useful
  for intro.
\item
  Cain \emph{et al.} (2008) : Cover in the form of caves and vegetation
  can be a significant advantage to thermoregulation in bighorn sheep
  \emph{Ovis canadensis mexicana}. \texttt{Used} temperature loggers in
  refugia and observations of sheep to posit utilisation. Ambient temps
  taken as proxy for body temps (``thermal load''). For intro or disc.
\end{enumerate}

\section{Refs}\label{refs}

\begin{center}\rule{0.5\linewidth}{\linethickness}\end{center}

\hypertarget{refs}{}
\hypertarget{ref-Aublet2009}{}
Aublet, J.-F., Festa-Bianchet, M., Bergero, D. \& Bassano, B. (2009).
Temperature constraints on foraging behaviour of male alpine ibex (capra
ibex) in summer. \emph{Oecologia}, 159, 237--247.

\hypertarget{ref-VANBEEST2012723}{}
Beest, F.M. van, Moorter, B.V. \& Milner, J.M. (2012).
Temperature-mediated habitat use and selection by a heat-sensitive
northern ungulate. \emph{Animal Behaviour}, 84, 723--735.

\hypertarget{ref-bennitt2014buffalo}{}
Bennitt, M.C.A.H., Emily AND Bonyongo. (2014). Habitat selection by
african buffalo (syncerus caffer) in response to landscape-level
fluctuations in water availability on two temporal scales. \emph{PLOS
ONE}, 9, 1--14.

\hypertarget{ref-ECY:ECY20119281648}{}
Boettiger, A.N., Wittemyer, G., Starfield, R., Volrath, F.,
Douglas-Hamilton, I. \& Getz, W.M. (2011). Inferring ecological and
behavioral drivers of african elephant movement using a linear filtering
approach. \emph{Ecology}, 92, 1648--1657.

\hypertarget{ref-bowyer2009thermal}{}
Bowyer, R.T. \& Kie, J.G. (2009). Thermal landscapes and resource
selection by black-tailed deer: Implications for large herbivores.
\emph{Calif Fish Game}, 95, 128--139.

\hypertarget{ref-cain2008responses}{}
Cain III, J.W., Krausman, P.R., Morgart, J.R., Jansen, B.D. \& Pepper,
M.P. (2008). Responses of desert bighorn sheep to removal of water
sources. \emph{Wildlife Monographs}, 1--32.

\hypertarget{ref-CAIN20081518}{}
Cain, J., Jansen, B., Wilson, R. \& Krausman, P. (2008). Potential
thermoregulatory advantages of shade use by desert bighorn sheep.
\emph{Journal of Arid Environments}, 72, 1518--1525.

\hypertarget{ref-cain_etal_2006}{}
Cain, J.W., Krausman, P.R., Rosenstock, S.S. \& Turner, J.C. (2006).
Mechanisms of thermoregulation and water balance in desert ungulates.
\emph{Wildlife Society Bulletin (1973-2006)}, 34, 570--581.

\hypertarget{ref-JZO:JZO848}{}
Cain, J.W., Owen-Smith, N. \& Macandza, V.A. (2012). The costs of
drinking: Comparative water dependency of sable antelope and zebra.
\emph{Journal of Zoology}, 286, 58--67.

\hypertarget{ref-Coughenour2008}{}
Coughenour, M.B. (2008). Causes and consequences of herbivore movement
in landscape ecosystems. In: \emph{Fragmentation in semi-arid and arid
landscapes: Consequences for human and natural systems} (eds. Galvin,
K.A., Reid, R.S., Jr, R.H.B. \& Hobbs, N.T.). Springer Netherlands,
Dordrecht, pp. 45--91.

\hypertarget{ref-cromhout2007ecology}{}
Cromhout, M. (2007). The ecology of the african buffalo in the eastern
kalahari region, south africa. PhD thesis..

\hypertarget{ref-de2008determinants}{}
De Beer, Y.-M. (2008). Determinants and consequences of elephant spatial
use in southern africa's arid savannas. PhD thesis..

\hypertarget{ref-Fuller159}{}
Fuller, A., Hetem, R.S., Maloney, S.K. \& Mitchell, D. (2014).
Adaptation to heat and water shortage in large, arid-zone mammals.
\emph{Physiology}, 29, 159--167.

\hypertarget{ref-Fuller2855}{}
Fuller, A., Kamerman, P.R., Maloney, S.K., Matthee, A., Mitchell, G. \&
Mitchell, D. (2005). A year in the thermal life of a free-ranging herd
of springbok antidorcas marsupialis. \emph{Journal of Experimental
Biology}, 208, 2855--2864.

\hypertarget{ref-giotto2015}{}
Giotto, N., Gerard, J.-F., Ziv, A., Bouskila, A. \& Bar-David, S.
(2015). Space-use patterns of the asiatic wild ass (equus hemionus):
Complementary insights from displacement, recursion movement and habitat
selection analyses. \emph{PLOS ONE}, 10, 1--21.

\hypertarget{ref-JEZ:JEZ389}{}
Hetem, R.S., Maloney, S.K., Fuller, A., Meyer, L.C. \& Mitchell, D.
(2007). Validation of a biotelemetric technique, using ambulatory
miniature black globe thermometers, to quantify thermoregulatory
behaviour in ungulates. \emph{Journal of Experimental Zoology Part A:
Ecological Genetics and Physiology}, 307A, 342--356.

\hypertarget{ref-hetem2012411}{}
Hetem, R.S., Strauss, W.M., Fick, L.G., Maloney, S.K., Meyer, L.C. \&
Shobrak, M. \emph{et al.} (2012). Activity re-assignment and
microclimate selection of free-living arabian oryx: Responses that could
minimise the effects of climate change on homeostasis? \emph{Zoology},
115, 411--416.

\hypertarget{ref-Hetem2010}{}
Hetem, R.S., Strauss, W.M., Fick, L.G., Maloney, S.K., Meyer, L.C.R. \&
Shobrak, M. \emph{et al.} (2010). Variation in the daily rhythm of body
temperature of free-living arabian oryx (oryx leucoryx): Does water
limitation drive heterothermy? \emph{Journal of Comparative Physiology
B}, 180, 1111--1119.

\hypertarget{ref-hirst1975ungulate}{}
Hirst, S.M. (1975). Ungulate-habitat relationships in a south african
woodland/savanna ecosystem. \emph{Wildlife Monographs}, 3--60.

\hypertarget{ref-JANE:JANE595}{}
Johnson, C.J., Parker, K.L., Heard, D.C. \& Gillingham, M.P. (2002).
Movement parameters of ungulates and scale-specific responses to the
environment. \emph{Journal of Animal Ecology}, 71, 225--235.

\hypertarget{ref-KINAHAN200747}{}
Kinahan, A., Pimm, S. \& Aarde, R. van. (2007). Ambient temperature as a
determinant of landscape use in the savanna elephant, loxodonta
africana. \emph{Journal of Thermal Biology}, 32, 47--58.

\hypertarget{ref-AJE:AJE1101}{}
Leggett, K. (2010). Daily and hourly movement of male desert-dwelling
elephants. \emph{African Journal of Ecology}, 48, 197--205.

\hypertarget{ref-loarie2009fences}{}
Loarie, S.R., Van Aarde, R.J. \& Pimm, S.L. (2009). Fences and
artificial water affect african savannah elephant movement patterns.
\emph{Biological conservation}, 142, 3086--3098.

\hypertarget{ref-lowe2010lack}{}
Lowe, S.J., Patterson, B.R. \& Schaefer, J.A. (2010). Lack of behavioral
responses of moose (alces alces) to high ambient temperatures near the
southern periphery of their range. \emph{Canadian Journal of Zoology},
88, 1032--1041.

\hypertarget{ref-Nowack2013}{}
Nowack, J., Wippich, M., Mzilikazi, N. \& Dausmann, K.H. (2013).
Surviving the cold, dry period in africa: Behavioral adjustments as an
alternative to heterothermy in the african lesser bushbaby (galago
moholi). \emph{International Journal of Primatology}, 34, 49--64.

\hypertarget{ref-owensmith2014coping}{}
Owen-Smith, N. \& Goodall, V. (2014). Coping with savanna seasonality:
Comparative daily activity patterns of african ungulates as revealed by
gps telemetry. \emph{Journal of Zoology}, 293, 181--191.

\hypertarget{ref-parmesan2006ecological}{}
Parmesan, C. (2006). Ecological and evolutionary responses to recent
climate change. \emph{Annu. Rev. Ecol. Evol. Syst.}, 37, 637--669.

\hypertarget{ref-rahimi2007movement}{}
Rahimi, S. \& Owen-Smith, N. (2007). Movement patterns of sable antelope
in the kruger national park from gps/gsm collars: A preliminary
assessment. \emph{South African Journal of Wildlife Research}, 37,
143--151.

\hypertarget{ref-redfern2002manipulating}{}
Redfern, J.V. (2002). Manipulating surface water availability to manage
herbivore distributions in the kruger national park, south africa.
PhD thesis. University of California, Berkeley.

\hypertarget{ref-ECY:ECY20038482092}{}
Redfern, J.V., Grant, R., Biggs, H. \& Getz, W.M. (2003). Surface-water
constraints on herbivore foraging in the kruger national park, south
africa. \emph{Ecology}, 84, 2092--2107.

\hypertarget{ref-schmidt2016ungulate}{}
Schmidt, N.M., Beest, F.M. van, Mosbacher, J.B., Stelvig, M., Hansen,
L.H. \& Nabe-Nielsen, J. \emph{et al.} (2016). Ungulate movement in an
extreme seasonal environment: Year-round movement patterns of
high-arctic muskoxen. \emph{Wildlife Biology}, 22, 253--267.

\hypertarget{ref-shrestha2012171}{}
Shrestha, A., Wieren, S. van, Langevelde, F. van, Fuller, A., Hetem, R.
\& Meyer, L. \emph{et al.} (2012). Body temperature variation of south
african antelopes in two climatically contrasting environments.
\emph{Journal of Thermal Biology}, 37, 171--178.

\hypertarget{ref-shrestha2012living}{}
Shrestha, A.K. (2012). \emph{Living on the edge: Physiological and
behavioural plasticity of african antelopes along a climatic gradient}.

\hypertarget{ref-Shrestha2014}{}
Shrestha, A.K., Wieren, S.E. van, Langevelde, F. van, Fuller, A., Hetem,
R.S. \& Meyer, L. \emph{et al.} (2014). Larger antelopes are sensitive
to heat stress throughout all seasons but smaller antelopes only during
summer in an african semi-arid environment. \emph{International Journal
of Biometeorology}, 58, 41--49.

\hypertarget{ref-Stelzner1988}{}
Stelzner, J.K. (1988). Thermal effects on movement patterns of yellow
baboons. \emph{Primates}, 29, 91--105.

\hypertarget{ref-TSHIPA201746}{}
Tshipa, A., Valls-Fox, H., Fritz, H., Collins, K., Sebele, L. \& Mundy,
P. \emph{et al.} (2017). Partial migration links local surface-water
management to large-scale elephant conservation in the world's largest
transfrontier conservation area. \emph{Biological Conservation}, 215,
46--50.

\hypertarget{ref-valls2015drink}{}
Valls Fox, H. (2015). To drink or not to drink? The influence of
resource availability on elephant foraging and habitat selection in a
semi-arid savanna.

\hypertarget{ref-walther2002ecological}{}
Walther, G.-R., Post, E., Convey, P., Menzel, A., Parmesan, C. \&
Beebee, T.J. \emph{et al.} (2002). Ecological responses to recent
climate change. \emph{Nature}, 416, 389--395.

\hypertarget{ref-Weissenbuxf6ck2012}{}
Weissenböck, N.M., Arnold, W. \& Ruf, T. (2012). Taking the heat:
Thermoregulation in asian elephants under different climatic conditions.
\emph{Journal of Comparative Physiology B}, 182, 311--319.

\end{document}
